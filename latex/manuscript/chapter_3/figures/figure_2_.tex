\begin{figure}
    \centering
    \begin{tikzpicture}[thick,scale=1.0, every node/.style={scale=1.0}]
        
    \begin{scope}[xshift=0.0cm]
        \node[inner sep=0pt] at (0, 0) {\includegraphics[scale=0.54]{chapter_3/figures/figure_2.pdf}};
        \begin{scope}[yshift=1.6cm, xshift=0.0cm]
            \node[inner sep=0pt] at (0, 0.05) {\includegraphics[width=0.9\textwidth]{chapter_3/figures/figure_2_legend.pdf}};
            \node[scale=0.7] at (-4.8, 0) {Fashion:COvsSH};
            \node[scale=0.7] at (-2.5, 0) {Fashion:SAvsBO};
            \node[scale=0.7] at (-0.25, 0) {Fashion:TOvsPU};
            \node[scale=0.7] at (1.8, 0) {MNIST:1vs7};
            \node[scale=0.7] at (3.75, 0) {MNIST:4vs9};
            \node[scale=0.7] at (5.65, 0) {MNIST:5vs6};
        \end{scope}
        \node[inner sep=0pt] at (-4.7, -1.6) {$\Risk_{\dTpert}(\MVQ)$};
        \node[inner sep=0pt] at (-1.45, -1.6){\footnotesize Bound of \Cref{chap:mv-robustness:eq:seeger-chromatic-T}};  
        \node[inner sep=0pt] at (1.9, -1.6) {$\RiskM_{\dTpert}(\MVQ)$};
        \node[inner sep=0pt] at (5.1, -1.6) {\footnotesize Bound of \Cref{chap:mv-robustness:eq:max-mcallester-T}};
        \node[inner sep=0pt,rotate=90] at (-6.95, 0.0) {$\RiskA_{\dT}(\MVQ)$};
    \end{scope}
       \end{tikzpicture}
    \caption[Visualization of the risk and bound values for ``$\text{Defense}{=}\text{Attack}$'']{
    Visualization of the risk and bound values for ``$\text{Defense}{=}\text{Attack}$'' when the set of voters is $\Hsigned$.
   Results obtained with the \PGDU, respectively \IFGSMU, defense are represented by a star $\bigstar$, respectively a circle $\newmoon$ 
   {\it   (reminder: $\RiskA_{\dT}(\MVQ)$ is computed with a \PGD, respectively \IFGSM, attack).}
   The dashed line corresponds to bisecting line $y{=}x$. 
    For $\Risk_{\dTpert}(\MVQ)$ and $\RiskM_{\dTpert}(\MVQ)$, the closer the datasets are to the bisecting line, the more accurate our relaxed risk is compared to the classical adversarial risk $\RiskA_{\dT}(\MVQ)$.
    For the bounds, the closer the datasets are to the bisecting line, the tighter the bound.
   }
    \label{chap:mv-robustness:fig:summarized-results}
\end{figure}